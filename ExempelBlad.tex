\documentclass[a4paper, twoside, titlepage]{blad}
\usepackage{amsmath,amsfonts,amssymb,graphicx} 
%amsmath används ganska ofta graphicx är till för att använda grafik

\usepackage{verbatim}
\usepackage[T1]{fontenc}        % används för att få svensk avstavning
\usepackage[swedish]{babel}     
\usepackage[latin1]{inputenc}   % och å, ä ,ö
\usepackage{moreverb}
\usepackage{xspace}
\usepackage{float}

%\setlength{\parindent}{0pt}     % tar bort indrag från stycken. Avslaget
%\setlength{\parskip}{3pt}       % Ändra så att stycken skiljs av
                                %blankrader. Avslaget
%\addtolength{\topmargin}{-0.8cm} %Minskar marginalerna litegrann
%\addtolength{\textheight}{0.8cm}

% Titel, författare etc.
\title{Ett Exempel på Sångblad}
\author{\includegraphics[width=.8 \textwidth ]{logo.eps}	}
\date{}                          %Ta bort kommentaren om du inte vill ha med datum.

\begin{document}
\pagenumbering{arabic}
\maketitle
\begin{sang}{Portos Visa}
Jag vill börja gasqua, var fan är min flaska?\\*
Vem i helvete stal min butelj?\\*
Skall törsten mig tvinga, en TT börja svinga\\*
Nej, för fan bara blunda och svälj\\*
Vilken smörja, får jag spörja\\*
Vem för fan tror att jag är en älg?\\*
Till England vi rider och sedan vad det lider\\*
Träffar vi välan på någon pub\\*
Och där skall vi festa, blott dricka av det bästa\\*
Utav Whisky och portvin, jag tänker gå hårt in\\*
För att prova på rubb och stubb
\end{sang}


\begin{sang}{Système International}
W kg m Wb s\\*
$\Omega$m T A rad\\*
Cd S N s\\*
$\Omega$ A m Lx dB\\*
$^\circ$C W/m$^2$\\*
J/kg H V C\\*
kg/m$^3$ mol\\*
m/s$^2$\\*
m/s$^2$\\*
F!
\end{sang}


\begin{sang}{Årskursenas hederssång}
Alla:   För det var i vår ungdoms fagraste vår,
Vi drack varandra till och vi sade "gutår".
Och alla så dricka vi nu Foo till! \\

Foo:    Och Foo säger inte nej därtill...

09 Falsklarm \\
08 F\o rspel \\
07 Fokus \\
06 Friskus \\
05 Flörtfrisk \\
04 FanFar \\
03 Fetvadd \\
02 Fusklapp \\
\end{sang}


\begin{sang}{O hemska lab}
O hemska lab, o grymma kval imorgon,\\*
Här sitter jag och förstår ingenting.\\*
Hela mitt inre fylls utav ett motstånd\\*
Emot eländig elektrisk mätteknik.\\*
Jag skulle nog behöva litet ledning,\\*
Här räcker inte min kapacitans.\\*
Kondensatorer och felvända dioder,\\*
O hemska lab, nu vill jag koppla af.\\*
O hemska lab, ty detta blir min graf!
\end{sang}


\begin{sang}{Jag har aldrig vart på snusen}
Jag har aldrig vart på snusen,\\*
Aldrig rökat en cigarr, haleluja\\*
Mina dygder äro tusen\\*
Inga syndiga laster jag har\\*
Jag har aldrig sett nå't naket\\*
Inte ens ett litet nyfött barn\\*
Mina blickar går mot taket\\*
Därmed undgår jag frestarens garn\\*
Haleluja, haleluja...

Bacchus spelar på gitarren\\*
Satan spelar på sitt handklaver\\*
Alla djävlar dansar tango\\*
Säg vad kan man väl önska sig mer?\\*
Jo, att alla bäckar vore brännvin\\*
Riddarfjärden full av bayerskt öl\\*
Konjak i varenda rännsten\\*
Och punsch i varenda vattenpöl\\*
Och mellanöl, och mellanöl... 
\end{sang}

\begin{sang}{Handelsvisan}
Vi vill aldrig gå på Handels,\\*
Aldrig tenta företagsekonomi.\\*
Deras IQ den e' Mandels\\*
Och förståndet, det har ju gjort sorti.\\*
Dom har jätteusla snören,\\*
Till sitt jätteusla draperi.\\*
Dom kan bara räkna ören,\\*
Hela skolan e' ett enda aperi!\\*
Handels är skit - Jag vill ej dit ....

Mammons pojkar är dom alla,\\*
Pappas flickor är dom likaså,\\*
Går och tror att dom är balla,\\*
Fastän dom inget alls ju förstå.\\*
Hela Handels borde rivas,\\*
Detta anser hela vårat lag.\\*
Då skulle Osquarulda trivas\\*
Uppå denna Handels ljuva domedag!\\*
Åh, vilket drag - på denna dag ....
\end{sang}

\begin{sang}{Fysikhatarvisan}
Jag vill inte gå på fysik\\*
aldrig tenta termometerdynamik\\*
Jag vill inte höra syntmusik\\*
inte festa som en tråkig mattegeek\\*
Vi ser ut som televerket\\*
i vår jättefula overall\\*
vi kan bara räkna kvarkar\\*
nu hyllar vi Data med en skål!\\*
Fysik är torrt - jag vill ju bort...

Einsteins pojkar är vi alla\\*
Handels flickor kan vi aldrig få\\*
går och tror att vi har ballar\\*
det får bli på egen hand om det ska gå\\*
Nu ska hela Sing-Sing rivas\\*
Arkitekt är med på Datas lag\\*
televerket ska fördrivas\\*
uppå konsulatets ljuva domedag\\*
Å nubbedrag - på denna dag...
\end{sang}


\begin{sang}{Konglig Fysiks Paradhymn}
Här på festen stiger åter glammet\\*
Sången börjar, tentan bortglömd är\\*
Lyss min strupe Du plågas utav dammet\\*
Frukta ej ty hjälpen är just här\\*
Ibland oss\\*
Höj pokalen, dess flöde känns som sammet\\*
Drick till det som Bacchi vapen lär.

Känn, o Osquar, känn hur blodet hettar\\*
Du Osqulda orsak till det är\\*
Timmar skrider och dygdens bojor lättar\\*
Fest och glädje kärleksflamman när\\*
Men minns att\\*
Blott ej synen en hungrig kärlek mättar\\*
Drick till det som Venus' vapen lär.

FYSIKER, gasqueropen de har skallat\\*
Likt musik från någon högre sfär\\*
Tentans piska för länge har oss vallat\\*
Trotsa den och studiernas misär\\*
Med lärdom\\*
Från de makter som ytterst har oss kallat\\*
Bacchus, Venus värdar hos oss är\\*
Och vänner\\*
Bacchi nektar ej Venus' flamma släcker\\*
SKÅL för det Fysiks skyddsgudar lär.
\end{sang}


\begin{sang}{O gamla klang och jubeltid}
O gamla klang och jubeltid\\*
Ditt minne skall förbliva\\*
Och än åt livets bistra strid\\*
Ett rosigt skimmer giva\\*
Snart tystnar allt vårt yra skämt\\*
Vår sång blir stum, vårt skratt förstämt\\*
O, jerum jerum jerum\\*
O, quae mutatio rerum

Var äro de som kunde allt\\*
Blott ej sin ära svika\\*
Som vora män av äkta halt\\*
Och världens herrar lika\\*
De drogo bort från vin och sång\\*
Till vardagslivets tråk och tvång\\*
O, jerum...

Den ene vetenskap och vett\\*
In i scholares mängder\\*
Den andre i sitt anlets svett\\*
På paragrafer vränger\\*
En plåstrar själen som är skral\\*
En lappar hop dess trasiga fodral\\*
O, jerum...

Men hjärtat i en sann student\\*
Kan ingen tid förfrysa\\*
Den glädjen eld som där han tänt\\*
Hans hela liv skall lysa\\*
Det gamla skalet brustit har\\*
Men kärnan finnes frisk dock kvar\\*
Och vad han än skall mista\\*
Den skall dock aldrig brista

Så sluten bröder fast vår krets\\*
Till glädjens värn och ära\\*
Trots allt vi tryggt och väl tillfreds\\*
Vår vänskap trohet svära\\*
Lyft bägarn högt och klinga vän\\*
De gamla gudar leva än\\*
Bland skålar och pokaler\\*
Bland skålar och pokaler
\end{sang}

\end{document}





